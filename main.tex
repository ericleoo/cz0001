\documentclass[a4paper,10pt]{article}
\usepackage[utf8]{inputenc}
\usepackage{setspace}
\usepackage{fullpage}
\usepackage{hyperref}
\usepackage{url}
\title{}
\author{}
\date{}
\begin{document}

%hellox
%\maketitle
%\doublespacing
%\section{Reminder}
%\begin{itemize}
%	\item Deadline: 20 March 2018
%	\item Pages: 8-15
%	\item Description: select  2, summarize, discuss why made  the  greatest  impact  to  Singapore’s  economic,  infrastructural  and  societal development since 1965
%\end{itemize}

\section{Content}
\doublespacing
\subsection{Singapore's First MRT System}

The Mass Rapid Transit, or MRT, is a rapid transit system forming the major component of the railway system in Singapore, spanning most of the city-state. The earliest section of the MRT, between Toa Payoh and Yio Chu Kang, opened on 7 November 1987. The network has since grown rapidly in accordance with Singapore's aim of developing a comprehensive rail network as the backbone of the public transport system in Singapore, with an average daily ridership of 3.031 million in 2015 (including the Light Rail Transit (LRT)), approximately 78\% of the bus network's 3.891 million in the same period.[2]

The MRT network encompasses 199.6 kilometres (124.0 mi) of route, with 119 stations in operation, on standard gauge. The fully automated Circle, Downtown and North East lines form the longest fully automated metro network in the world.[3][4] The lines are built by the Land Transport Authority, a statutory board of the Government of Singapore, which allocates operating concessions to the profit-based corporations, SMRT Corporation and SBS Transit. These operators also run bus and taxi services, thus facilitating full integration of public transport services. The MRT is complemented by a small number of local LRT networks in Bukit Panjang, Sengkang and Punggol that link MRT stations with HDB public housing estates.[5]

\subsubsection{Economic}
The MRT is one of the keys to Singapore's rapid economic growth into a first world country. Among the neighbouring countries in South East Asia, Singapore is the first to implement such transportation system that spans island-wide.

An economic analysis\cite{fouracre1993mass} conducted in 1993 shows that the MRT provides Singapoa cvre with 20.5\% economic internal rate of return. The beneficiaries of the train are largely existing public transport users. Together their time savings account for almost 75\% of the benefits.

\subsubsection{Infrastructural}

The MRT is essential for Singapore's public transportation system. The existence of such public transit system that spans most of the country is one of the factors that made Singapore possible to successfully restrain the use of cars. Since many people prefer to travel by MRT, traffic congestion problems rarely ever happen in Singapore.

\subsubsection{Societal}

Today, over three million people use the MRT daily - more than half of the entire population \cite{ltaridership}. The keys to MRT's success is mainly its reliability, effectiveness, and affordability. Since it is not only affordable, but also reliable and effective, people from any social classes can benefit from the MRT, unlike other public transports such as cabs.


\subsection{Next Generation Nationwide Broadband Network}

Built as a project under the Intelligent Nation 2015 (iN2015) masterplan, the Next Generation Nationwide Broadband Network (NGNBN) is Singapore's Fibre-to-Anywhere wired network. The NGNBN makes it possible to competitively price broadband network speeds of up to 1Gbps.

It is expected for the ultra high-speed broadband to provide lots of benefits to users and enterprises almost anywhere in Singapore since the network spans nationwide with 95\% of homes and businesses in Singapore covered already. The broadband network is offered by service providers and is typically available in different plans; each is customized to target different types of users.

\subsubsection{Economic}

\subsubsection{Infrastructural}

\subsubsection{Societal}


\bibliography{bibs.bib} 
\bibliographystyle{ieeetr}

\end{document}
