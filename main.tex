\documentclass[a4paper,10pt]{article}
\usepackage[utf8]{inputenc}
\usepackage{setspace}
\usepackage{fullpage}
\title{}
\author{}
\date{}
\begin{document}

%\maketitle
%\doublespacing
%\section{Reminder}
%\begin{itemize}
%	\item Deadline: 20 March 2018
%	\item Pages: 8-15
%	\item Description: select  2, summarize, discuss why made  the  greatest  impact  to  Singapore’s  economic,  infrastructural  and  societal development since 1965
%\end{itemize}

\section{Content}
\doublespacing
\subsection{Singapore's First MRT System}
The Mass Rapid Transit, or MRT, is a rapid transit system forming the major component of the railway system in Singapore, spanning most of the city-state. The earliest section of the MRT, between Toa Payoh and Yio Chu Kang, opened on 7 November 1987. The network has since grown rapidly in accordance with Singapore's aim of developing a comprehensive rail network as the backbone of the public transport system in Singapore, with an average daily ridership of 3.031 million in 2015 (including the Light Rail Transit (LRT)), approximately 78\% of the bus network's 3.891 million in the same period.[2]

The MRT network encompasses 199.6 kilometres (124.0 mi) of route, with 119 stations in operation, on standard gauge. The fully automated Circle, Downtown and North East lines form the longest fully automated metro network in the world.[3][4] The lines are built by the Land Transport Authority, a statutory board of the Government of Singapore, which allocates operating concessions to the profit-based corporations, SMRT Corporation and SBS Transit. These operators also run bus and taxi services, thus facilitating full integration of public transport services. The MRT is complemented by a small number of local LRT networks in Bukit Panjang, Sengkang and Punggol that link MRT stations with HDB public housing estates.[5]

\subsubsection{Economic}
The MRT is one of the keys to Singapore's economic growth into a first world country. The system is reliable, useful, and efficient. The existence of such public transit system results in increased business

An economic analysis\cite{fouracre1993mass} shows that the MRT provides Singapore with 20.5\% economic internal rate of return.

\subsubsection{Infrastructural}

\subsubsection{Societal}

\subsection{Jurong Island}
Jurong Island is formed from the reclamation of seven southern islands: Pulau Seraya, Pulau Ayer Merbau, Pulau Sakra, Pulau Pesek Kecil, Pulau Pesek, Pulau Ayer Chawan and Pulau Merlimau.[1] In 1991, Jurong Town Corporation (JTC) was appointed the agent of the Jurong Island project.[2] By October 2000, a sum of S\$7 billion had been invested in phases one, two and three of the reclamation of the island.[3]

The idea of building Jurong Island was conceived by the government as early as the 1980s.[4] The government had envisioned the island to be a premier regional chemical hub, with the petrochemical sector poised to reduce the economy’s dependence on electronics manufacturing.[5] In the early 1990s, the world’s major chemical companies were persuaded by the Economic Development Board (EDB) to invest millions of dollars into plots of sea that would later be reclaimed to become Jurong Island.[6]
 
In February 2000, then Deputy Prime Minister Lee Hsien Loong announced at the official opening of the Jurong Island Road Link, which connects the island to mainland Singapore, that JTC planned to establish a major chemical logistics hub on Jurong Island to serve the fast-growing petrochemical industry.  At the time, the island was already home to some 55 petroleum and chemical companies with total combined assets worth S\$21 billion.[7] The hub would be equipped with its own berths, jetties and other marine facilities, as well as provide logistics services like storage tanks, chemical warehouses, tank filling, cleaning and maintenance, drumming and waste treatment facilities.[8]

Jurong Island was officially opened by then Prime Minister Goh Chok Tong on 14 October 2000. In his speech, Goh congratulated the EDB and JTC on the “Herculean achievement”, and pointed out that Jurong Island is a good example of ingenuity in creating significant synergy and economies of scale for the tenants, and offering them better value for money.[9] By the time Jurong Island officially opened, there were more than 60 petroleum, petrochemical, specialty chemical and supporting companies established on the island, with investments of more than S\$20 billion. It was envisaged that by 2010, the island would accommodate 150 companies with investments of S\$40 billion, and employing some 15,000 people.[10]`

\subsubsection{Economic}

\subsubsection{Infrastructural}

\subsubsection{Societal}

\bibliography{bibs.bib} 
\bibliographystyle{ieeetr}

\end{document}
